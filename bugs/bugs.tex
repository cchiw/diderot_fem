% ---- ETD Document Class and Useful Packages ---- %
\documentclass{article}
\usepackage[margin=0.5in]{geometry}
\usepackage{subfigure,epsfig,amsfonts}
\usepackage{natbib}
\usepackage{amsmath}
\usepackage{amssymb}
\usepackage{amsthm}
\usepackage{listings} % for code 
\usepackage{enumitem} % for creating lists
\usepackage{lineno, blindtext} % line numbers
\usepackage{xstring} %support if/else inside commands
\usepackage{longtable}
\usepackage{color}
\usepackage{multicol}
\usepackage{url}
%%%%%%%%%%%%%%%%%%%%%%%%%%%%%%%%%%%%
%\input{../defs/packs}
%\input{../defs/dissertation_format}
%\input{../defs/def2} %generic input
\newcommand{\mynote}[3]{\textcolor{#3}{\small\textbf{\textsf{{#1}: {#2}}}}}
\newcommand{\CC}[1]{\mynote{cc}{#1}{magenta}}


\newcommand{\fontit}[1]{\textit{#1}}
\newcommand{\checkname}[0]{\textit{DATm}}
%%%%%%%%%%%%%%%%%%%%%%%%%%%%%%%%%%%%
 \newcommand{\testA}[0]{$\textit{Test}_{differential}$}
 \newcommand{\testB}[0]{$\textit{Test}_{equality}$}
 \newcommand{\testC}[0]{$\textit{Test}_{property}$}
  \newcommand{\testD}[0]{$\textit{Test}_{vis}$}
  \newcommand{\DFEM}[0]{\textbf{Diderot-Fire}}
    \newcommand{\DNrrd}[0]{\textbf{Diderot-Nrrd}}
%%%%%%%%%%%%%%%%%%%%%%%%%%%%%%%%%%%%
\begin{document}
Quick table to describe bugs, location and status.\\
Bug type: compilation (C) and numerical (N) bug.\\


\begin{figure}
\begin{tabular}{|l|l|} 
\hline
\multicolumn{2}{|l|}{\textbf{results not correct}}\\
\hline
\# &Description \\
\hline
 F2   & magnitude of a field\\
 F10   &  Hessian of addition \\
   F16    & sampling outside mesh  \\
\hline
\hline
\multicolumn{2}{|l|}{Compilation issue \textbf{ Missing support for operators on FE fields}}\\
\hline
\# & Description \\
\hline
 F3 & Inner/cross product \\
F12 & Tensor slicing\\
   F5 &Inside test\\
F17  &   Division (r,v)\\
  F20  & Double dot product\\
\hline  
\hline
\multicolumn{2}{|l|}{ compilation issue \textbf{ Internal representation of FE fields}}\\
\hline
\# & Description \\
\hline
 F6   & Summation not handled \\
  F8  & Differential indices are constants\\
    F8  &components indices are constants\\
\hline    
\hline
\multicolumn{2}{|l|}{compilation issue \textbf{Generated code for point-wise eval}}\\
\hline
\# & Description \\
\hline
 F1 &Converting types\\
F7  & Converting types\\
 F4   &Multiple creation of functions  \\
F14   & ``\\
F19   & ``\\
   F9  &   Weird indexing tensors\\
 F15  &  Type error for helper function\\
\hline  
\hline
\multicolumn{2}{|l|}{\textbf{Other}}\\
\hline
\# & Description \\
\hline
  F1   &run-time\\
\hline
\end{tabular}
\caption{Organization of bugs}
\label{bugs}
\end{figure}




\begin{tabular}{|l|ll|lll|}
\hline
\# & Status & Folder & Description \\
\hline
 F3 &Closed&b3  & Inner/cross product  \\\
  F12 &Closed & b11 & Tensor slicing\\
   F5 &OPEN&b5& Inside test\\
F17 &OPEN &   & Division (r,v)\\
  F20 &OPEN & b20  & Double dot product\\
 F6 &Closed &b6  & Summation not handled \\
  F8 &Closed &b8 &Differential indices are constants\\
    F8 &Closed &b8 &components ndices are constants\\
 F1 &Closed& b1& Converting types\\
F7 &Closed& b6  & Converting types\\
 F4 &Closed&b4 &Multiple creation of functions with the same name \\
F14 &OPEN & b14 &Multiple creation of functions with the same name\\
F19 &OPEN & b19 &Multiple creation of functions with the same name\\
   F9 &OPEN&b9 & Weird indexing tensors\\
 F15 &OPEN&b5 & Type error for helper function\\
  F16 &CLOSED & b16  & sampling outside mesh and derivative find cell  \\
 F2 &Closed &b2  &\\
  F13 &OPEN & b12 \& b11  &weird issue at run-time\\
 F10 &OPEN &b10 &  Hessian of addition (unknown)?\\

\hline
\end{tabular}
\\


\section{Questions}
\begin{itemize}[noitemsep]
\item Is F1 the same as F7?
\item  Is F4 the same as F14?  
\item How to describe F9 (weird indexing tensors) issue?
\item How to describe F13 (weird run-time) issue?
\item Should F16 be a code generation error or a logic error? Was there an issue translating the tree ir to generated code or something else?
\item Hessian of addition still fails even when we kill polynomial order (k) value.
\end{itemize}



\section{Comparisons}
Can the bug be found with different approaches?\\
\begin{tabular}{|l|llll|}
\hline
\# & \testA{} & \testB{} & \testC{} & \testD{}\\
\hline
\end{tabular}



\section{Bugs in FEMprime branch}

\begin{description}[noitemsep]
\item[issue]- top level description
\item[computation]- operators and arguments
\item[output]-terminal output (if helpful)
\item[solution]- how was it solved 
\item[details]- of problem
\item[versions] svn version  scope (error-solved)
\end{description}

Sections with * indicates the bug still needs to be better understood.
\subsection{F1*}
\begin{description}[noitemsep]
\item[issue]- Code generation issue when converting types
\item[computation]-$|(\nabla(F0))|;$\
\item[output].\\
\begin{lstlisting}[mathescape=true]
ex1.cxx:798:16: error: no viable conversion from 'ex1::tensor_ref_2' to 'double'
        double l_probe_l_4_22 = makeEval_UnitSquareMesh_Lagrange_2_1(
               ^                ~~~~~~~~~~~~~~~~~~~~~~~~~~~~~~~~~~~~~
***rtn:compile __p_o25_o6_t1_tN_tN__l2
\end{lstlisting}
\item[solution]-?
\item[details]
\end{description}


\subsection{F2*}
\begin{description}[noitemsep]
\item[issue] Numerical error  when taking the norm.
\item[computation] $normalize(\nabla(F0))$
\item[output]
\item[solution]
\item[details]
\end{description}

 
\subsection{F3} 
\begin{description}[noitemsep]
\item[issue] Missing support for inner/cross product on ofields
\item[computation]
\item[output]
\item[solution] Add ofields (inner product) to typechecker
\item[details] Can not take inner-product of ofields 
\end{description}



\subsection{F4} 
\begin{description}[noitemsep]
\item[issue] Multiple creation of functions with the same name
\item[computation] gradient of a field
\item[output] terminal\\
\begin{lstlisting}[mathescape=true]
ex1.cxx:423:17: error: redefinition of 'helpEvalBasis_UnitSquareMesh_Lagrange_2'
inline double * helpEvalBasis_UnitSquareMesh_Lagrange_2(const double *k...
\end{lstlisting}
\item[solution]
\item[details]
\end{description}




\subsection{F5*}
\begin{description}[noitemsep]
\item[issue]
\item[computation]
\item[output] terminal.\\
\begin{lstlisting}[mathescape=true]
3pow TIλ(T0[])<(T0)^2>HighToMid.expandOp: error converting InsideFEM<3>
uncaught exception Bind [nonexhaustive binding failure]
  raised at common/phase-timer.sml:78.57-78.59
  raised at high-to-mid/high-to-mid.sml:203.105-203.107
  raised at high-to-mid/buil
  \end{lstlisting}
\item[solution]
\item[details] Inside Error
\end{description}


  
\subsection{F6}
\begin{description}[noitemsep]
\item[issue] Translation in compiler EIN IR for Summation not handled
\item[computation]
\item[output] terminal.\\
\begin{lstlisting}[mathescape=true]
3HighToMid.expandEINAPP: error converting out051A = λ(F0[3],FNCSPACE1,FNCSPACE2,T3[3])<Probe(BuildFEM(T0)_1[2]∇_{i0,i1}),T3)> (FF004AB, VF004AE, pathF004B1, pos0421)
uncaught exception Subscript [subscript out of bounds]
  raised at common/phase-timer.sml:78.57-78.59
  raised at high-to-mid/high-to-mid.sml:216.7-216.9
  raised at Basis/Implementation/list.sml:78.35-78.44
make: *** [ex1.o] Error 1
cp: ex1.cxx: No such file or directory
cp: ex1.cxx: No such file or directory
***rtn:compile __p_o24_o1_t2_t2__l2
	 -: trace(hessian)
	-_F_s_d3 |p_o24_o1_t2_t2__l2
	rtn:compile 
	  \end{lstlisting}
\item[solution]
\item[details]
Summation in a single term not handled correctly
\end{description}


\subsection{F7*} 
\begin{description}[noitemsep]
\item[issue]conversion of types done incorrectly 
\item[computation]
\item[output] terminal.\\
\begin{lstlisting}[mathescape=true]
	  ex1.cxx:870:6: error: no type named 'tensor_ref_3_3' in namespace 'ex1'
ex1::tensor_ref_3_3 s_makeEval_UnitCubeMesh_P_4_2(NodeTy nodes, newposTy b, coordTy c,int cell,MappTy nM, FloatMapTy pM){
~~~~~^
ex1.cxx:876:14: error:
\end{lstlisting}
\item[solution]
\item[details]
\end{description}
\subsection{F8}
\begin{description}[noitemsep]
\item[issue]Differential indices are constants
\item[computation]det(concat2)
\item[output] terminal.
\begin{lstlisting}[mathescape=true]
λ(F0[2],FNCSPACE1,FNCSPACE2,T3[3])<Probe(BuildFEM(T0_{'0'})_1[2]),T3)>
\end{lstlisting}
\item[details] Unhandled cases when using constant indices.Constant indices in field components
\item[solution]
\end{description}

\subsection{F9*}
\begin{description}[noitemsep]
\item[issue] Weird indexing tensors
\item[computation]
\item[output] terminal.\\
\begin{lstlisting}[mathescape=true]
ex1.cxx:850:20: error: subscripted value is not an array, pointer, or vector
        H[0][0][0] = H0[0][0];
                     ~~~~~^~
ex1.cxx:851:20: error: subscripted value is not an array, pointer, or vector
        H[0][1][0] = H0[1][0];
\end{lstlisting}
\item[solution]
\item[details]
\end{description}

\subsection{F10}
\begin{description}[noitemsep]
\item[issue]	Issue unknown 
\item[computation]
\item[output] terminal.\\
\begin{lstlisting}[mathescape=true]     	
-p_o8_o24_t12_t2__l2 hessian(addition)| F_s_d3,F_s_d3| 
Rst: Z-3 RD  max diff: 51.5896 sumdiff: 13.9962 67.4956% c:24.8443995169280 o:76.43399
    	\end{lstlisting}
\item[solution]
\item[details]
\end{description}

	\subsection{F11*}
    	
	\subsection{F12}
	\begin{description}[noitemsep]
\item[issue]
\item[computation]
\item[output]
\item[solution]
\item[details]
\end{description}
		\begin{lstlisting}[mathescape=true]
uncaught exception Fail [Fail: unknown type]
  raised at common/phase-timer.sml:78.57-78.59
  raised at driver/main.sml:84.76-84.79
  raised at typechecker/check-expr.sml:611.47-611.66
make: *** [ex1.o] Error 1
cp: ex1.cxx: No such file or directory
cp: ex1.cxx: No such file or directory


***rtn:compile __p_o23_o29_t2_tN_tN__l2
	 -: slicev0(grad)
	-_F_s_d3 |p_o23_o29_t2_tN_tN__l2
	rtn:compile 
	\end{lstlisting}
\subsection{F13}
\begin{description}[noitemsep]
\item[issue] Unknown
\item[computation]
\item[output]
Weird allocation error\\
\begin{lstlisting}[mathescape=true]
python(24510,0x7fff796c2300) malloc: *** error for object 0x7ff48bbf7800: pointer being freed was not allocated
*** set a breakpoint in malloc_error_break to debug
[Charisees-MacBook-Air:24510] *** Process received signal ***
[Charisees-MacBook-Air:24510] Signal: Abort trap: 6 (6)
[Charisees-MacBook-Air:24510] Signal code:  (0)
[Charisees-MacBook-Air:24510] [ 0] 0   libsystem_platform.dylib            0x00007fff88cd3f1a _sigtramp + 26
[Charisees-MacBook-Air:24510] [ 1] 0   ???                                 0x0000000000000000 0x0 + 0
[Charisees-MacBook-Air:24510] [ 2] 0   libsystem_c.dylib                   0x00007fff88d439a3 abort + 129
[Charisees-MacBook-Air:24510] [ 3] 0   libsystem_malloc.dylib              0x00007fff8d8941cb free + 428
	\end{lstlisting}
\item[solution]
\item[details]
Error when creating a vector fields in python
\end{description}

\subsection{F14}
\begin{description}[noitemsep]
\item[issue]Multiple creation of functions with the same name
\item[computation]
\item[output]
\begin{lstlisting}[mathescape=true]
	0_tensor[2] compositionl,ex1.cxx:777:15: error: redefinition of 's_makeEval_UnitCubeMesh_Lagrange_4_'
inline double s_makeEval_UnitCubeMesh_Lagrange_4_(NodeTy nodes, newposT...
Makefile
	\end{lstlisting}
\item[solution]
\item[details]
\end{description}
	
\subsection{F15}
\begin{description}[noitemsep]
\item[issue]
\item[computation]
\item[output]teriminal\\.
\begin{lstlisting}[mathescape=true]
observ.cxx:919:2: error: no matching function for call to
      'jIs_UnitSquareMesh_P_2'
        jIs_UnitSquareMesh_P_2(J,nM,pM,cell);
        ^~~~~~~~~~~~~~~~~~~~~~
observ.cxx:831:14: note: candidate function not viable: no known conversion from
      'double [2][2]' to 'double (*)[3]' for 1st argument
inline void *jIs_UnitSquareMesh_P_2(double J[3][3],MappTy nM, FloatMapT...
	\end{lstlisting}
\item[solution]
\item[details]

\end{description}


\subsection{F16}
\begin{description}[noitemsep]
\item[issue]
\item[computation]
\item[output]
\item[solution]
\item[details] Sampling at the point [0,9.45187e+06] led to a find cell error, which was handled correctly, but the derivative code did not handle this case and segfaulted. The reason it sampled to far away was a mistake in FATm. 
\end{description}






\end{document}

