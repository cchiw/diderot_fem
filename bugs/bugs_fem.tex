\section{Bugs in FEMprime branch}

\begin{description}
\item{FC1} Code generation issue when converting types\\
$|(\nabla(F0))|;$

\begin{lstlisting}[mathescape=true]
ex1.cxx:798:16: error: no viable conversion from 'ex1::tensor_ref_2' to 'double'
        double l_probe_l_4_22 = makeEval_UnitSquareMesh_Lagrange_2_1(
               ^                ~~~~~~~~~~~~~~~~~~~~~~~~~~~~~~~~~~~~~
***rtn:compile __p_o25_o6_t1_tN_tN__l2
\end{lstlisting}
\item{FN2}
Numerical error  when taking the norm. Possible mistake causes from mini-merge and new operators.
$normalize(\nabla(F0))$ also   norm(hessian)\\

 
\item{FC3} Can not take inner-product of fields with different continuity.
\begin{lstlisting}[mathescape=true]
Type error (adding ofields-check typechecker)
***rtn:compile __p_o25_o11_t1_t7_tN__l2
***rtn:compile __p_o25_o12_t1_t7_tN__l2
\end{lstlisting}
\item{FC4} Multiple creation of functions with the same name (gradient of a field).
\begin{lstlisting}[mathescape=true]
ex1.cxx:423:17: error: redefinition of 'helpEvalBasis_UnitSquareMesh_Lagrange_2'
inline double * helpEvalBasis_UnitSquareMesh_Lagrange_2(const double *k...
\end{lstlisting}

\item{FC5}
\begin{lstlisting}[mathescape=true]
3pow TIλ(T0[])<(T0)^2>HighToMid.expandOp: error converting InsideFEM<3>
uncaught exception Bind [nonexhaustive binding failure]
  raised at common/phase-timer.sml:78.57-78.59
  raised at high-to-mid/high-to-mid.sml:203.105-203.107
  raised at high-to-mid/buil
  \end{lstlisting}
  
\item{FC6}



 python pde.py 4 24 1 2 2
\begin{lstlisting}[mathescape=true]
3HighToMid.expandEINAPP: error converting out051A = λ(F0[3],FNCSPACE1,FNCSPACE2,T3[3])<Probe(BuildFEM(T0)_1[2]∇_{i0,i1}),T3)> (FF004AB, VF004AE, pathF004B1, pos0421)
uncaught exception Subscript [subscript out of bounds]
  raised at common/phase-timer.sml:78.57-78.59
  raised at high-to-mid/high-to-mid.sml:216.7-216.9
  raised at Basis/Implementation/list.sml:78.35-78.44
make: *** [ex1.o] Error 1
cp: ex1.cxx: No such file or directory
cp: ex1.cxx: No such file or directory
***rtn:compile __p_o24_o1_t2_t2__l2
	 -: trace(hessian)
	-_F_s_d3 |p_o24_o1_t2_t2__l2
	rtn:compile 
	  \end{lstlisting}
\item{FC7}  
	  next error
\begin{lstlisting}[mathescape=true]
	  ex1.cxx:870:6: error: no type named 'tensor_ref_3_3' in namespace 'ex1'
ex1::tensor_ref_3_3 s_makeEval_UnitCubeMesh_P_4_2(NodeTy nodes, newposTy b, coordTy c,int cell,MappTy nM, FloatMapTy pM){
~~~~~^
ex1.cxx:876:14: error:
\end{lstlisting}
\item{FC8}
Unhandled cases when using constant indices
-Differential indices are constants
- Constant indices in field components
\begin{lstlisting}[mathescape=true]
	λ(F0[2],FNCSPACE1,FNCSPACE2,T3[3])<Probe(BuildFEM(T0_{'0'})_1[2]),T3)>
	det(concat2) |22_3- 
	\end{lstlisting}
	
	\item[FC9]
	\begin{lstlisting}[mathescape=true]
                ^
ex1.cxx:850:20: error: subscripted value is not an array, pointer, or vector
        H[0][0][0] = H0[0][0];
                     ~~~~~^~
ex1.cxx:851:20: error: subscripted value is not an array, pointer, or vector
        H[0][1][0] = H0[1][0];
        	\end{lstlisting}
	\item[FT10] 
	Issue unknown       	
   	\begin{lstlisting}[mathescape=true]     	-p_o8_o24_t12_t2__l2 hessian(addition)| F_s_d3,F_s_d3| 
		Rst: Z-3 RD  max diff: 51.5896 sumdiff: 13.9962 67.4956% c:24.8443995169280 o:76.43399
    	\end{lstlisting}
    	
	\item[FC11]
		\begin{lstlisting}[mathescape=true]
uncaught exception Fail [Fail: unknown type]
  raised at common/phase-timer.sml:78.57-78.59
  raised at driver/main.sml:84.76-84.79
  raised at typechecker/check-expr.sml:611.47-611.66
make: *** [ex1.o] Error 1
cp: ex1.cxx: No such file or directory
cp: ex1.cxx: No such file or directory


***rtn:compile __p_o23_o29_t2_tN_tN__l2
	 -: slicev0(grad)
	-_F_s_d3 |p_o23_o29_t2_tN_tN__l2
	rtn:compile 
	\end{lstlisting}
\item[FT12]
Weird allocation error\
	    				\begin{lstlisting}[mathescape=true]
    		Error when creating a vector fields in python. Syntax probably wrong.
    		
    		python(24510,0x7fff796c2300) malloc: *** error for object 0x7ff48bbf7800: pointer being freed was not allocated
*** set a breakpoint in malloc_error_break to debug
[Charisees-MacBook-Air:24510] *** Process received signal ***
[Charisees-MacBook-Air:24510] Signal: Abort trap: 6 (6)
[Charisees-MacBook-Air:24510] Signal code:  (0)
[Charisees-MacBook-Air:24510] [ 0] 0   libsystem_platform.dylib            0x00007fff88cd3f1a _sigtramp + 26
[Charisees-MacBook-Air:24510] [ 1] 0   ???                                 0x0000000000000000 0x0 + 0
[Charisees-MacBook-Air:24510] [ 2] 0   libsystem_c.dylib                   0x00007fff88d439a3 abort + 129
[Charisees-MacBook-Air:24510] [ 3] 0   libsystem_malloc.dylib              0x00007fff8d8941cb free + 428
	\end{lstlisting}

	    		\item[CF14]
\begin{lstlisting}[mathescape=true]
	0_tensor[2] compositionl,ex1.cxx:777:15: error: redefinition of 's_makeEval_UnitCubeMesh_Lagrange_4_'
inline double s_makeEval_UnitCubeMesh_Lagrange_4_(NodeTy nodes, newposT...
              ^
	\end{lstlisting}
	
		\item[CF15]
		error supporting inner product and outer product.
              \item[FCN16] Sampling at the point [0,9.45187e+06] led to a find cell error, which was handled correctly, but the derivative code did not handle this case and segfaulted. The reason it sampled to far away was a mistake in FATm. 
\end{description}


