\section{Bugs in FEMprime branch}

\begin{description}[noitemsep]
\item[issue]- top level description
\item[computation]- operators and arguments
\item[output]-terminal output (if helpful)
\item[solution]- how was it solved 
\item[details]- of problem
\item[versions] svn version  scope (error-solved)
\end{description}

Sections with * indicates the bug still needs to be better understood.
\subsection{F1*}
\begin{description}[noitemsep]
\item[issue]- Code generation issue when converting types
\item[computation]-$|(\nabla(F0))|;$\
\item[output].\\
\begin{lstlisting}[mathescape=true]
ex1.cxx:798:16: error: no viable conversion from 'ex1::tensor_ref_2' to 'double'
        double l_probe_l_4_22 = makeEval_UnitSquareMesh_Lagrange_2_1(
               ^                ~~~~~~~~~~~~~~~~~~~~~~~~~~~~~~~~~~~~~
***rtn:compile __p_o25_o6_t1_tN_tN__l2
\end{lstlisting}
\item[solution]-?
\item[details]
\end{description}


\subsection{F2*}
\begin{description}[noitemsep]
\item[issue] Numerical error  when taking the norm.
\item[computation] $normalize(\nabla(F0))$
\item[output]
\item[solution]
\item[details]
\end{description}

 
\subsection{F3} 
\begin{description}[noitemsep]
\item[issue] Missing support for inner/cross product on ofields
\item[computation]
\item[output]
\item[solution] Add ofields (inner product) to typechecker
\item[details] Can not take inner-product of ofields 
\end{description}



\subsection{F4} 
\begin{description}[noitemsep]
\item[issue] Multiple creation of functions with the same name
\item[computation] gradient of a field minus another field
\item[output] terminal\\
\begin{lstlisting}[mathescape=true]
ex1.cxx:423:17: error: redefinition of 'helpEvalBasis_UnitSquareMesh_Lagrange_2'
inline double * helpEvalBasis_UnitSquareMesh_Lagrange_2(const double *k...
\end{lstlisting}
\item[solution] Fixed in r5413: the solution was to realize that the gradient of a field and a field incorrectly both generated common functions and stop this.
\item[details]
\end{description}




\subsection{F5*}
\begin{description}[noitemsep]
\item[issue]
\item[computation]
\item[output] terminal.\\
\begin{lstlisting}[mathescape=true]
3pow TIλ(T0[])<(T0)^2>HighToMid.expandOp: error converting InsideFEM<3>
uncaught exception Bind [nonexhaustive binding failure]
  raised at common/phase-timer.sml:78.57-78.59
  raised at high-to-mid/high-to-mid.sml:203.105-203.107
  raised at high-to-mid/buil
  \end{lstlisting}
\item[solution]
\item[details] Inside Error
\end{description}


  
\subsection{F6}
\begin{description}[noitemsep]
\item[issue] Translation in compiler EIN IR for Summation not handled
\item[computation]
\item[output] terminal.\\
\begin{lstlisting}[mathescape=true]
3HighToMid.expandEINAPP: error converting out051A = λ(F0[3],FNCSPACE1,FNCSPACE2,T3[3])<Probe(BuildFEM(T0)_1[2]∇_{i0,i1}),T3)> (FF004AB, VF004AE, pathF004B1, pos0421)
uncaught exception Subscript [subscript out of bounds]
  raised at common/phase-timer.sml:78.57-78.59
  raised at high-to-mid/high-to-mid.sml:216.7-216.9
  raised at Basis/Implementation/list.sml:78.35-78.44
make: *** [ex1.o] Error 1
cp: ex1.cxx: No such file or directory
cp: ex1.cxx: No such file or directory
***rtn:compile __p_o24_o1_t2_t2__l2
	 -: trace(hessian)
	-_F_s_d3 |p_o24_o1_t2_t2__l2
	rtn:compile 
	  \end{lstlisting}
\item[solution]
\item[details]
Summation in a single term not handled correctly
\end{description}


\subsection{F7*} 
\begin{description}[noitemsep]
\item[issue]conversion of types done incorrectly 
\item[computation]
\item[output] terminal.\\
\begin{lstlisting}[mathescape=true]
	  ex1.cxx:870:6: error: no type named 'tensor_ref_3_3' in namespace 'ex1'
ex1::tensor_ref_3_3 s_makeEval_UnitCubeMesh_P_4_2(NodeTy nodes, newposTy b, coordTy c,int cell,MappTy nM, FloatMapTy pM){
~~~~~^
ex1.cxx:876:14: error:
\end{lstlisting}
\item[solution]
\item[details]
\end{description}
\subsection{F8}
\begin{description}[noitemsep]
\item[issue]Differential indices are constants
\item[computation]det(concat2)
\item[output] terminal.
\begin{lstlisting}[mathescape=true]
λ(F0[2],FNCSPACE1,FNCSPACE2,T3[3])<Probe(BuildFEM(T0_{'0'})_1[2]),T3)>
\end{lstlisting}
\item[details] Unhandled cases when using constant indices.Constant indices in field components
\item[solution]
\end{description}

\subsection{F9*}
\begin{description}[noitemsep]
\item[issue] Weird indexing tensors
\item[computation]
\item[output] terminal.\\
\begin{lstlisting}[mathescape=true]
ex1.cxx:850:20: error: subscripted value is not an array, pointer, or vector
        H[0][0][0] = H0[0][0];
                     ~~~~~^~
ex1.cxx:851:20: error: subscripted value is not an array, pointer, or vector
        H[0][1][0] = H0[1][0];
\end{lstlisting}
\item[solution]
\item[details]
\end{description}

\subsection{F10}
\begin{description}[noitemsep]
\item[issue]	Issue unknown 
\item[computation]
\item[output] terminal.\\
\begin{lstlisting}[mathescape=true]     	
-p_o8_o24_t12_t2__l2 hessian(addition)| F_s_d3,F_s_d3| 
Rst: Z-3 RD  max diff: 51.5896 sumdiff: 13.9962 67.4956% c:24.8443995169280 o:76.43399
    	\end{lstlisting}
\item[solution]
\item[details]
\end{description}

	\subsection{F11*}
    	
	\subsection{F12}
	\begin{description}[noitemsep]
\item[issue]
\item[computation]
\item[output]
\item[solution]
\item[details]
\end{description}
		\begin{lstlisting}[mathescape=true]
uncaught exception Fail [Fail: unknown type]
  raised at common/phase-timer.sml:78.57-78.59
  raised at driver/main.sml:84.76-84.79
  raised at typechecker/check-expr.sml:611.47-611.66
make: *** [ex1.o] Error 1
cp: ex1.cxx: No such file or directory
cp: ex1.cxx: No such file or directory


***rtn:compile __p_o23_o29_t2_tN_tN__l2
	 -: slicev0(grad)
	-_F_s_d3 |p_o23_o29_t2_tN_tN__l2
	rtn:compile 
	\end{lstlisting}
\subsection{F13}
\begin{description}[noitemsep]
\item[issue] Unknown
\item[computation]
\item[output]
Weird allocation error\\
\begin{lstlisting}[mathescape=true]
python(24510,0x7fff796c2300) malloc: *** error for object 0x7ff48bbf7800: pointer being freed was not allocated
*** set a breakpoint in malloc_error_break to debug
[Charisees-MacBook-Air:24510] *** Process received signal ***
[Charisees-MacBook-Air:24510] Signal: Abort trap: 6 (6)
[Charisees-MacBook-Air:24510] Signal code:  (0)
[Charisees-MacBook-Air:24510] [ 0] 0   libsystem_platform.dylib            0x00007fff88cd3f1a _sigtramp + 26
[Charisees-MacBook-Air:24510] [ 1] 0   ???                                 0x0000000000000000 0x0 + 0
[Charisees-MacBook-Air:24510] [ 2] 0   libsystem_c.dylib                   0x00007fff88d439a3 abort + 129
[Charisees-MacBook-Air:24510] [ 3] 0   libsystem_malloc.dylib              0x00007fff8d8941cb free + 428
	\end{lstlisting}
\item[solution]
\item[details]
Error when creating a vector fields in python
\end{description}

\subsection{F14}
\begin{description}[noitemsep]
\item[issue]Multiple creation of functions with the same name
\item[computation]
\item[output]
\begin{lstlisting}[mathescape=true]
	0_tensor[2] compositionl,ex1.cxx:777:15: error: redefinition of 's_makeEval_UnitCubeMesh_Lagrange_4_'
inline double s_makeEval_UnitCubeMesh_Lagrange_4_(NodeTy nodes, newposT...
Makefile
	\end{lstlisting}
\item[solution]
\item[details]
\end{description}
	
\subsection{F15}
\begin{description}[noitemsep]
\item[issue]
\item[computation]
\item[output]teriminal\\.
\begin{lstlisting}[mathescape=true]
observ.cxx:919:2: error: no matching function for call to
      'jIs_UnitSquareMesh_P_2'
        jIs_UnitSquareMesh_P_2(J,nM,pM,cell);
        ^~~~~~~~~~~~~~~~~~~~~~
observ.cxx:831:14: note: candidate function not viable: no known conversion from
      'double [2][2]' to 'double (*)[3]' for 1st argument
inline void *jIs_UnitSquareMesh_P_2(double J[3][3],MappTy nM, FloatMapT...
	\end{lstlisting}
\item[solution]
\item[details]

\end{description}


\subsection{F16}
\begin{description}[noitemsep]
\item[issue] Accidental sampling outside of a cell leads to a segfault in derivative code
\item[computation] Any level of differentiation and a sampling outside of the mesh will cause this error. 
\item[output] MPI reads out a segfault. 
\item[solution] Add check in the derivative code for outside of the mesh sampling
\item[details] Sampling at the point [0,9.45187e+06] led to a find cell error, which was handled correctly, but the derivative code did not handle this case and segfaulted. The reason it sampled to far away was a mistake in FATm. 
\end{description}

\subsection{F21}
\begin{description}[noitemsep]
\item[issue] Numerical error. 
\item[computation] Trace
\item[output] terminal\\
\begin{lstlisting}[mathescape=true]
***rtn:terrible__p_o0_o6_t16_tN_tN__l2
	 -: trace(none)
	-
-p_o0_o6_t16_tN_tN__l2 trace(none)| F_m2x2_d2| 
		Rst: Z-3 RD  max diff: 7.486 sumdiff: 4.3919 172.2781% c:-3.14070000000000 o:4.3453002
	rtn:terrible
	\end{lstlisting}
\item[solution] splitting summation of probe. then creating own operator. shifting upper bound +1.
\item[details] Summation with indices in field component
\end{description}

\subsection{F22}
\begin{lstlisting}[mathescape=true]
-p_o0_o7_t16_tN__l2 det(none)| F_m2x2_d2| Rst: V-0 RA
	\end{lstlisting}
Field component has two constant indices.
cutting function in float was not organizing indices correctly.

 python fem.py 3 0 7 23  (again with -3d case)
 

\subsection{F23}
\begin{lstlisting}[mathescape=true]
le __p_o0_o8_t16_tN__l2
	 -: inverse(none)
	 
1ex1.cxx:838:41: error: no matching function for call to 's_2_2makeEval_UnitSquareMesh_P_2_'
        tensor_ref_2_2 l_probe_l_4_24 = s_2_2makeEval_UnitSquareMesh_P_2_(l_node_23, l_newpos_22, l_coordinates_14);
                                        ^~~~~~~~~~~~~~~~~~~~~~~~~~~~~~~~~
ex1.cxx:655:21: note: candidate function not viable: no known conversion from 'ex1::tensor_ref_2' to 'newposTy'
      (aka 'double *') for 2nd argument
ex1::tensor_ref_2_2 s_2_2makeEval_UnitSquareMesh_P_2_(NodeTy nodes, newposTy b, coordTy c ){
                    ^
ex1.cxx:853:43: error: use of undeclared identifier 's_2makeEval_UnitSquareMesh_P_2_'; did you mean
      's_2_2makeEval_UnitSquareMesh_P_2_'?
            tensor_ref_2 l_probe_l_4_33 = s_2makeEval_UnitSquareMesh_P_2_(l_node_23, l_newpos_22, l_coordinates_14);
                                          ^~~~~~~~~~~~~~~~~~~~~~~~~~~~~~~
                                          s_2_2makeEval_UnitSquareMesh_P_2_
	\end{lstlisting}


\subsection{F24}

\begin{description}
\item[Issue] Subtraction args are switched
\item[Computation] Subtracting one field from another
\item[Output]
  \begin{lstlisting}
    t__p_o0_o10_t14_t14__l2

observed data from femprime/ [[0.1088, -3.8392, -4.5563998], [-1.4155999, -2.8296001, -5.9981999], [3.6296, -7.6163998, -2.1338], [2.5702, -4.7568002, -1.7131], [1.8088, -5.2392001, -3.1064], [1.0176001, -4.3983998, -3.6528001], [1.5876, -5.9884, -3.9828]]

correct data from python[[-0.108800000000000, 3.83920000000000, 4.55640000000000], [1.41560000000000, 2.82960000000000, 5.99820000000000], [-3.62960000000000, 7.61640000000000, 2.13380000000000], [-2.57020000000000, 4.75680000000000, 1.71310000000000], [-1.80880000000000, 5.23920000000000, 3.10640000000000], [-1.01760000000000, 4.39840000000000, 3.65280000000000], [-1.58760000000000, 5.98840000000000, 3.98280000000000]]

 positions[[0.46, 0.38], [0.41, 0.14], [0.82, 0.63], [0.41, 0.87], [0.56, 0.58], [0.48, 0.52], [0.72, 0.43]]

Params(2)

-p_o0_o10_t14_t14__l2 subtraction(none)| F_v3_d2,F_v3_d2| 
		Rst: Z-3 RD  max diff: 15.2328 sumdiff: 6.8524 200.0% c:7.61640000000000 o:-7.6163998


  \end{lstlisting}
\end{description}

\subsection{F25}

\begin{description}
\item[Issue] Code gen error
\item[Computation] Derivative of a tensor field i.e if you pass a tensor field from Firedrake and takes its derivative in Diderot
\item[Output]  It crashes warning of a double free. I don't have this on hand. Try the commit r5403 to see it. I forget the commit that fixed it
.\item[solution] Stopped any functions from delete memory allocated for newpos.
\item[details] Evaluation functions assumed that they were the last ones to get a bit of memory (the memory for the new position; allocated in the translate coordinates process) and so they delete it too soon.

\end{description}

\subsection{F26}

\begin{description}
\item[Issue] Code gen or floating point problem
\item[Computation] Regular pointwise evaluation
\item[Output] Numerically incorrect results on various combinations of points and FE data
\item[Solution] Kick the can by using doubles
\end{description}